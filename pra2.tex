% Options for packages loaded elsewhere
\PassOptionsToPackage{unicode}{hyperref}
\PassOptionsToPackage{hyphens}{url}
%
\documentclass[
]{article}
\usepackage{amsmath,amssymb}
\usepackage{lmodern}
\usepackage{iftex}
\ifPDFTeX
  \usepackage[T1]{fontenc}
  \usepackage[utf8]{inputenc}
  \usepackage{textcomp} % provide euro and other symbols
\else % if luatex or xetex
  \usepackage{unicode-math}
  \defaultfontfeatures{Scale=MatchLowercase}
  \defaultfontfeatures[\rmfamily]{Ligatures=TeX,Scale=1}
\fi
% Use upquote if available, for straight quotes in verbatim environments
\IfFileExists{upquote.sty}{\usepackage{upquote}}{}
\IfFileExists{microtype.sty}{% use microtype if available
  \usepackage[]{microtype}
  \UseMicrotypeSet[protrusion]{basicmath} % disable protrusion for tt fonts
}{}
\makeatletter
\@ifundefined{KOMAClassName}{% if non-KOMA class
  \IfFileExists{parskip.sty}{%
    \usepackage{parskip}
  }{% else
    \setlength{\parindent}{0pt}
    \setlength{\parskip}{6pt plus 2pt minus 1pt}}
}{% if KOMA class
  \KOMAoptions{parskip=half}}
\makeatother
\usepackage{xcolor}
\IfFileExists{xurl.sty}{\usepackage{xurl}}{} % add URL line breaks if available
\IfFileExists{bookmark.sty}{\usepackage{bookmark}}{\usepackage{hyperref}}
\hypersetup{
  pdftitle={Tipología y ciclo de vida de los datos: Práctica 2},
  pdfauthor={Jorge Marchán Gutiérrez; Rafael Jiménez Sarmentero},
  hidelinks,
  pdfcreator={LaTeX via pandoc}}
\urlstyle{same} % disable monospaced font for URLs
\usepackage[margin=1in]{geometry}
\usepackage{color}
\usepackage{fancyvrb}
\newcommand{\VerbBar}{|}
\newcommand{\VERB}{\Verb[commandchars=\\\{\}]}
\DefineVerbatimEnvironment{Highlighting}{Verbatim}{commandchars=\\\{\}}
% Add ',fontsize=\small' for more characters per line
\usepackage{framed}
\definecolor{shadecolor}{RGB}{48,48,48}
\newenvironment{Shaded}{\begin{snugshade}}{\end{snugshade}}
\newcommand{\AlertTok}[1]{\textcolor[rgb]{1.00,0.81,0.69}{#1}}
\newcommand{\AnnotationTok}[1]{\textcolor[rgb]{0.50,0.62,0.50}{\textbf{#1}}}
\newcommand{\AttributeTok}[1]{\textcolor[rgb]{0.80,0.80,0.80}{#1}}
\newcommand{\BaseNTok}[1]{\textcolor[rgb]{0.86,0.64,0.64}{#1}}
\newcommand{\BuiltInTok}[1]{\textcolor[rgb]{0.80,0.80,0.80}{#1}}
\newcommand{\CharTok}[1]{\textcolor[rgb]{0.86,0.64,0.64}{#1}}
\newcommand{\CommentTok}[1]{\textcolor[rgb]{0.50,0.62,0.50}{#1}}
\newcommand{\CommentVarTok}[1]{\textcolor[rgb]{0.50,0.62,0.50}{\textbf{#1}}}
\newcommand{\ConstantTok}[1]{\textcolor[rgb]{0.86,0.64,0.64}{\textbf{#1}}}
\newcommand{\ControlFlowTok}[1]{\textcolor[rgb]{0.94,0.87,0.69}{#1}}
\newcommand{\DataTypeTok}[1]{\textcolor[rgb]{0.87,0.87,0.75}{#1}}
\newcommand{\DecValTok}[1]{\textcolor[rgb]{0.86,0.86,0.80}{#1}}
\newcommand{\DocumentationTok}[1]{\textcolor[rgb]{0.50,0.62,0.50}{#1}}
\newcommand{\ErrorTok}[1]{\textcolor[rgb]{0.76,0.75,0.62}{#1}}
\newcommand{\ExtensionTok}[1]{\textcolor[rgb]{0.80,0.80,0.80}{#1}}
\newcommand{\FloatTok}[1]{\textcolor[rgb]{0.75,0.75,0.82}{#1}}
\newcommand{\FunctionTok}[1]{\textcolor[rgb]{0.94,0.94,0.56}{#1}}
\newcommand{\ImportTok}[1]{\textcolor[rgb]{0.80,0.80,0.80}{#1}}
\newcommand{\InformationTok}[1]{\textcolor[rgb]{0.50,0.62,0.50}{\textbf{#1}}}
\newcommand{\KeywordTok}[1]{\textcolor[rgb]{0.94,0.87,0.69}{#1}}
\newcommand{\NormalTok}[1]{\textcolor[rgb]{0.80,0.80,0.80}{#1}}
\newcommand{\OperatorTok}[1]{\textcolor[rgb]{0.94,0.94,0.82}{#1}}
\newcommand{\OtherTok}[1]{\textcolor[rgb]{0.94,0.94,0.56}{#1}}
\newcommand{\PreprocessorTok}[1]{\textcolor[rgb]{1.00,0.81,0.69}{\textbf{#1}}}
\newcommand{\RegionMarkerTok}[1]{\textcolor[rgb]{0.80,0.80,0.80}{#1}}
\newcommand{\SpecialCharTok}[1]{\textcolor[rgb]{0.86,0.64,0.64}{#1}}
\newcommand{\SpecialStringTok}[1]{\textcolor[rgb]{0.80,0.58,0.58}{#1}}
\newcommand{\StringTok}[1]{\textcolor[rgb]{0.80,0.58,0.58}{#1}}
\newcommand{\VariableTok}[1]{\textcolor[rgb]{0.80,0.80,0.80}{#1}}
\newcommand{\VerbatimStringTok}[1]{\textcolor[rgb]{0.80,0.58,0.58}{#1}}
\newcommand{\WarningTok}[1]{\textcolor[rgb]{0.50,0.62,0.50}{\textbf{#1}}}
\usepackage{graphicx}
\makeatletter
\def\maxwidth{\ifdim\Gin@nat@width>\linewidth\linewidth\else\Gin@nat@width\fi}
\def\maxheight{\ifdim\Gin@nat@height>\textheight\textheight\else\Gin@nat@height\fi}
\makeatother
% Scale images if necessary, so that they will not overflow the page
% margins by default, and it is still possible to overwrite the defaults
% using explicit options in \includegraphics[width, height, ...]{}
\setkeys{Gin}{width=\maxwidth,height=\maxheight,keepaspectratio}
% Set default figure placement to htbp
\makeatletter
\def\fps@figure{htbp}
\makeatother
\setlength{\emergencystretch}{3em} % prevent overfull lines
\providecommand{\tightlist}{%
  \setlength{\itemsep}{0pt}\setlength{\parskip}{0pt}}
\setcounter{secnumdepth}{-\maxdimen} % remove section numbering
\ifLuaTeX
  \usepackage{selnolig}  % disable illegal ligatures
\fi

\title{Tipología y ciclo de vida de los datos: Práctica 2}
\author{Jorge Marchán Gutiérrez \and Rafael Jiménez Sarmentero}
\date{mayo 2022}

\begin{document}
\maketitle

{
\setcounter{tocdepth}{2}
\tableofcontents
}
\begin{center}\rule{0.5\linewidth}{0.5pt}\end{center}

\hypertarget{descripciuxf3n-del-dataset.-por-quuxe9-es-importante-y-quuxe9-preguntaproblema-pretende-responder}{%
\section{Descripción del dataset. ¿Por qué es importante y qué
pregunta/problema pretende
responder?}\label{descripciuxf3n-del-dataset.-por-quuxe9-es-importante-y-quuxe9-preguntaproblema-pretende-responder}}

\begin{center}\rule{0.5\linewidth}{0.5pt}\end{center}

El dataset elegido para la realización de la práctica ha sido el de
\href{https://www.kaggle.com/competitions/titanic/overview}{Titanic} que
contiene una serie de datos sobre los pasajeros del Titanic, entre otras
cosas, si finalmente sobrevivieron o no, los datos se dividen en varios
ficheros \texttt{train.csv} y \texttt{test.csv}, además de un tercer
fichero \texttt{gender\_submission.csv} que para la realizacion de esta
práctica no es necesario, ya que es un ejemplo de fichero de envio para
la competición, de Kaggle. A nosotros nos interesa el fichero de
\texttt{train.csv}, sobre el cual vamos a realizar las tareas de
limpieza y análisis.

Con este dataset se podrían encontrar relaciones entre supervivencia y
edad, o supervivencia y género, entre otras, o se podría utilizar para
entrenar un modelo capaz de predecir, si una persona con unas
características determinadas sobrevivió al accidente o no.

\begin{Shaded}
\begin{Highlighting}[]
\NormalTok{data }\OtherTok{\textless{}{-}} \FunctionTok{read.csv}\NormalTok{(}\StringTok{"./input\_files/train.csv"}\NormalTok{, }\AttributeTok{header =} \ConstantTok{TRUE}\NormalTok{, }\AttributeTok{stringsAsFactors =} \ConstantTok{FALSE}\NormalTok{)}
\FunctionTok{dim}\NormalTok{(data)}
\end{Highlighting}
\end{Shaded}

\begin{verbatim}
## [1] 891  12
\end{verbatim}

\begin{Shaded}
\begin{Highlighting}[]
\FunctionTok{head}\NormalTok{(data)}
\end{Highlighting}
\end{Shaded}

\begin{verbatim}
##   PassengerId Survived Pclass
## 1           1        0      3
## 2           2        1      1
## 3           3        1      3
## 4           4        1      1
## 5           5        0      3
## 6           6        0      3
##                                                  Name    Sex Age SibSp Parch
## 1                             Braund, Mr. Owen Harris   male  22     1     0
## 2 Cumings, Mrs. John Bradley (Florence Briggs Thayer) female  38     1     0
## 3                              Heikkinen, Miss. Laina female  26     0     0
## 4        Futrelle, Mrs. Jacques Heath (Lily May Peel) female  35     1     0
## 5                            Allen, Mr. William Henry   male  35     0     0
## 6                                    Moran, Mr. James   male  NA     0     0
##             Ticket    Fare Cabin Embarked
## 1        A/5 21171  7.2500              S
## 2         PC 17599 71.2833   C85        C
## 3 STON/O2. 3101282  7.9250              S
## 4           113803 53.1000  C123        S
## 5           373450  8.0500              S
## 6           330877  8.4583              Q
\end{verbatim}

Podemos observar que el dataset contiene 891 filas y 12 atributos, a
continuación vamos a ver los tipos de atributos y su significado

\begin{Shaded}
\begin{Highlighting}[]
\FunctionTok{str}\NormalTok{(data)}
\end{Highlighting}
\end{Shaded}

\begin{verbatim}
## 'data.frame':    891 obs. of  12 variables:
##  $ PassengerId: int  1 2 3 4 5 6 7 8 9 10 ...
##  $ Survived   : int  0 1 1 1 0 0 0 0 1 1 ...
##  $ Pclass     : int  3 1 3 1 3 3 1 3 3 2 ...
##  $ Name       : chr  "Braund, Mr. Owen Harris" "Cumings, Mrs. John Bradley (Florence Briggs Thayer)" "Heikkinen, Miss. Laina" "Futrelle, Mrs. Jacques Heath (Lily May Peel)" ...
##  $ Sex        : chr  "male" "female" "female" "female" ...
##  $ Age        : num  22 38 26 35 35 NA 54 2 27 14 ...
##  $ SibSp      : int  1 1 0 1 0 0 0 3 0 1 ...
##  $ Parch      : int  0 0 0 0 0 0 0 1 2 0 ...
##  $ Ticket     : chr  "A/5 21171" "PC 17599" "STON/O2. 3101282" "113803" ...
##  $ Fare       : num  7.25 71.28 7.92 53.1 8.05 ...
##  $ Cabin      : chr  "" "C85" "" "C123" ...
##  $ Embarked   : chr  "S" "C" "S" "S" ...
\end{verbatim}

Los atributos que encontramos son: * \textbf{PassengerId}: Es el
identificador interno del pasajero, de tipo entero * \textbf{Survived}:
Es un valor de tipo entero que nos indica si el pasajero ha sobrevivido
o no (0 o 1) * \textbf{Pclass}: El tipo de billete que ha adquirido el
pasajero, tipo entero (1 = Primera, 2 = Segunda, 3 = Tercera) *
\textbf{Name}: El nombre del pasajero, tipo char * \textbf{Sex}: El
género del pasajero, tipo char (male o female) * \textbf{Age}: La edad
del pasajero, tipo number * \textbf{SibSp}: El numero de hermanos y
conyuges que hay abordo en el Titanic, tipo entero * \textbf{Parch}: El
número de padres e hijos que hay abordo en el Titanic, tipo entero *
\textbf{Ticket}: El identificador del billete, tipo char *
\textbf{Fare}: El precio del billete, tipo number * \textbf{Cabin}: El
código del camarote, tipo char * \textbf{Embarked}: El puerto donde
embarco el pasajero, tipo char (C = Cherbourg, Q = Queenstown, S =
Southampton)

\begin{center}\rule{0.5\linewidth}{0.5pt}\end{center}

\hypertarget{integraciuxf3n-y-selecciuxf3n-de-los-datos-de-interuxe9s-a-analizar.-puede-ser-el-resultado-de-adicionar-diferentes-datasets-o-una-subselecciuxf3n-uxfatil-de-los-datos-originales-en-base-al-objetivo-que-se-quiera-conseguir.}{%
\section{Integración y selección de los datos de interés a analizar.
Puede ser el resultado de adicionar diferentes datasets o una
subselección útil de los datos originales, en base al objetivo que se
quiera
conseguir.}\label{integraciuxf3n-y-selecciuxf3n-de-los-datos-de-interuxe9s-a-analizar.-puede-ser-el-resultado-de-adicionar-diferentes-datasets-o-una-subselecciuxf3n-uxfatil-de-los-datos-originales-en-base-al-objetivo-que-se-quiera-conseguir.}}

\begin{center}\rule{0.5\linewidth}{0.5pt}\end{center}

\begin{center}\rule{0.5\linewidth}{0.5pt}\end{center}

\hypertarget{limpieza-de-los-datos.}{%
\section{Limpieza de los datos.}\label{limpieza-de-los-datos.}}

\begin{center}\rule{0.5\linewidth}{0.5pt}\end{center}

\hypertarget{los-datos-contienen-ceros-o-elementos-vacuxedos-gestiona-cada-uno-de-estos-casos.}{%
\subsection{¿Los datos contienen ceros o elementos vacíos? Gestiona cada
uno de estos
casos.}\label{los-datos-contienen-ceros-o-elementos-vacuxedos-gestiona-cada-uno-de-estos-casos.}}

\hypertarget{identifica-y-gestiona-los-valores-extremos.}{%
\subsection{Identifica y gestiona los valores
extremos.}\label{identifica-y-gestiona-los-valores-extremos.}}

\begin{center}\rule{0.5\linewidth}{0.5pt}\end{center}

\hypertarget{anuxe1lisis-de-los-datos.}{%
\section{Análisis de los datos.}\label{anuxe1lisis-de-los-datos.}}

\begin{center}\rule{0.5\linewidth}{0.5pt}\end{center}

\hypertarget{selecciuxf3n-de-los-grupos-de-datos-que-se-quieren-analizarcomparar-p.-e.-si-se-van-a-comparar-grupos-de-datos-cuuxe1les-son-estos-grupos-y-quuxe9-tipo-de-anuxe1lisis-se-van-a-aplicar}{%
\subsection{Selección de los grupos de datos que se quieren
analizar/comparar (p.~e., si se van a comparar grupos de datos, ¿cuáles
son estos grupos y qué tipo de análisis se van a
aplicar?)}\label{selecciuxf3n-de-los-grupos-de-datos-que-se-quieren-analizarcomparar-p.-e.-si-se-van-a-comparar-grupos-de-datos-cuuxe1les-son-estos-grupos-y-quuxe9-tipo-de-anuxe1lisis-se-van-a-aplicar}}

\hypertarget{comprobaciuxf3n-de-la-normalidad-y-homogeneidad-de-la-varianza.}{%
\subsection{Comprobación de la normalidad y homogeneidad de la
varianza.}\label{comprobaciuxf3n-de-la-normalidad-y-homogeneidad-de-la-varianza.}}

\hypertarget{aplicaciuxf3n-de-pruebas-estaduxedsticas-para-comparar-los-grupos-de-datos.-en-funciuxf3n-de-los-datos-y-el-objetivo-del-estudio-aplicar-pruebas-de-contraste-de-hipuxf3tesis-correlaciones-regresiones-etc.-aplicar-al-menos-tres-muxe9todos-de-anuxe1lisis-diferentes.}{%
\subsection{Aplicación de pruebas estadísticas para comparar los grupos
de datos. En función de los datos y el objetivo del estudio, aplicar
pruebas de contraste de hipótesis, correlaciones, regresiones, etc.
Aplicar al menos tres métodos de análisis
diferentes.}\label{aplicaciuxf3n-de-pruebas-estaduxedsticas-para-comparar-los-grupos-de-datos.-en-funciuxf3n-de-los-datos-y-el-objetivo-del-estudio-aplicar-pruebas-de-contraste-de-hipuxf3tesis-correlaciones-regresiones-etc.-aplicar-al-menos-tres-muxe9todos-de-anuxe1lisis-diferentes.}}

\begin{center}\rule{0.5\linewidth}{0.5pt}\end{center}

\hypertarget{representaciuxf3n-de-los-resultados-a-partir-de-tablas-y-gruxe1ficas.-este-apartado-se-puede-responder-a-lo-largo-de-la-pruxe1ctica-sin-necesidad-de-concentrar-todas-las-representaciones-en-este-punto-de-la-pruxe1ctica.}{%
\section{Representación de los resultados a partir de tablas y gráficas.
Este apartado se puede responder a lo largo de la práctica, sin
necesidad de concentrar todas las representaciones en este punto de la
práctica.}\label{representaciuxf3n-de-los-resultados-a-partir-de-tablas-y-gruxe1ficas.-este-apartado-se-puede-responder-a-lo-largo-de-la-pruxe1ctica-sin-necesidad-de-concentrar-todas-las-representaciones-en-este-punto-de-la-pruxe1ctica.}}

\begin{center}\rule{0.5\linewidth}{0.5pt}\end{center}

\begin{center}\rule{0.5\linewidth}{0.5pt}\end{center}

\hypertarget{resoluciuxf3n-del-problema.-a-partir-de-los-resultados-obtenidos-cuuxe1les-son-las-conclusiones-los-resultados-permiten-responder-al-problema}{%
\section{Resolución del problema. A partir de los resultados obtenidos,
¿cuáles son las conclusiones? ¿Los resultados permiten responder al
problema?}\label{resoluciuxf3n-del-problema.-a-partir-de-los-resultados-obtenidos-cuuxe1les-son-las-conclusiones-los-resultados-permiten-responder-al-problema}}

\begin{center}\rule{0.5\linewidth}{0.5pt}\end{center}

\end{document}
